\documentclass[11pt]{article} 
\usepackage[utf8]{inputenc}

%%% PAGE DIMENSIONS
\usepackage{geometry}
\geometry{a4paper}

\usepackage{graphicx} % support the \includegraphics command and options

% \usepackage[parfill]{parskip} % Activate to begin paragraphs with an empty line rather than an indent

%%% PACKAGES
\usepackage{float}
\usepackage{booktabs}
\usepackage{array}
\usepackage{paralist}
\usepackage{verbatim}
\usepackage{subfig} 
\usepackage{fancyhdr}
\usepackage{amsmath}
\usepackage{amssymb}

\pagestyle{fancy} % options: empty , plain , fancy
\renewcommand{\headrulewidth}{0pt} % customise the layout...
\lhead{}\chead{}\rhead{}
\lfoot{}\cfoot{\thepage}\rfoot{}

\usepackage{sectsty}
\allsectionsfont{\sffamily\mdseries\upshape}

\usepackage{listings}
\usepackage{color}

\definecolor{mygreen}{rgb}{0,0.6,0}
\definecolor{mygray}{rgb}{0.5,0.5,0.5}
\definecolor{mymauve}{rgb}{0.58,0,0.82}

\lstset{ 
  backgroundcolor=\color{white},   % choose the background color; you must add \usepackage{color} or \usepackage{xcolor}; should come as last argument
  basicstyle=\footnotesize,        % the size of the fonts that are used for the code
  breakatwhitespace=false,         % sets if automatic breaks should only happen at whitespace
  breaklines=true,                 % sets automatic line breaking
  captionpos=b,                    % sets the caption-position to bottom
  commentstyle=\color{mygreen},    % comment style
  firstnumber=1000,                % start line enumeration with line 1000
  frame=single,	                   % adds a frame around the code
  keepspaces=true,                 % keeps spaces in text, useful for keeping indentation of code (possibly needs columns=flexible)
  keywordstyle=\color{blue},       % keyword style
  language=Python,                 % the language of the code
  numbers=left,                    % where to put the line-numbers; possible values are (none, left, right)
  numbersep=5pt,                   % how far the line-numbers are from the code
  numberstyle=\tiny\color{mygray}, % the style that is used for the line-numbers
  rulecolor=\color{black},         % if not set, the frame-color may be changed on line-breaks within not-black text (e.g. comments (green here))
  showspaces=false,                % show spaces everywhere adding particular underscores; it overrides 'showstringspaces'
  showstringspaces=false,          % underline spaces within strings only
  showtabs=true,                   % show tabs within strings adding particular underscores
  stepnumber=1,                    % the step between two line-numbers. If it's 1, each line will be numbered
  stringstyle=\color{mymauve},     % string literal style
  tabsize=2,	                   % sets default tabsize to 2 spaces
  title=\lstname                   % show the filename of files included with \lstinputlisting; also try caption instead of title
}


\title{Práctica 4: Entrenamiento de Redes Neuronales}
\author{Ana Martín Sánchez, Nicolás Pastore Burgos}
\date{21/09/2021} 

\begin{document}
\maketitle

\section{Descripción de la práctica}

 En esta práctica, se pedía implementar la función de coste de una red neuronal para un conjunto de datos de entrenamiento. Posteriormente, se debía implementar el gradiente para esa red neuronal y, una vez comprobada su corrección, se pedía entrenar la red neuronal para obtener los valores óptimos de Theta1 y Theta2.
 
 Para ello, se utilizan los mismo datos de entrenamiento que en la práctica anterior: un conjunto de 500 imágenes de números escritos a mano, en el que cada imagen (de 20x20 píxeles) se representa como una matriz de 20x20 números, donde cada número indica la intensidad en escala de grises del píxel.

\section{Solución propuesta}

\subsection{Resultados obtenidos}

\subsubsection {Parte 1}

 En esta parte, teníamos que implementar una función de coste para una red neuronal, utilizando un algoritmo de propagación hacia delante. Con nuestra implementación, obtuvimos un coste de 0.287629.

Posteriormente, añadimos el término de regularización al coste, con lo que el coste subió hasta 0.384470.


\subsubsection {Parte 2}

 En esta segunda parte, necesitábamos calcular el gradiente de una red neuronal de tres o más capas. Para hacerlo, implementamos una función de retro-propagación. Primero, ejecutamos una propagación hacia delante para calcular la salida de la red, y posteriormente se ejecuta la retro-propagación, para calcular cuánto contribuye cada nodo al error total. Al hacerlo, obtuvimos un error de AAAAAAAAAAAAAAAAAAAAAAAAAAAAAAAAAA.

\newpage
\subsection{Implementación}

\lstinputlisting[language=Python]{main.py}
%\lstinputlisting[language=Python]{checkNNGradients.py}


\end{document}
